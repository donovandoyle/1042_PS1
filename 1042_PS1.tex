\documentclass[]{article}
\usepackage{lmodern}
\usepackage{amssymb,amsmath}
\usepackage{ifxetex,ifluatex}
\usepackage{fixltx2e} % provides \textsubscript
\ifnum 0\ifxetex 1\fi\ifluatex 1\fi=0 % if pdftex
  \usepackage[T1]{fontenc}
  \usepackage[utf8]{inputenc}
\else % if luatex or xelatex
  \ifxetex
    \usepackage{mathspec}
  \else
    \usepackage{fontspec}
  \fi
  \defaultfontfeatures{Ligatures=TeX,Scale=MatchLowercase}
\fi
% use upquote if available, for straight quotes in verbatim environments
\IfFileExists{upquote.sty}{\usepackage{upquote}}{}
% use microtype if available
\IfFileExists{microtype.sty}{%
\usepackage{microtype}
\UseMicrotypeSet[protrusion]{basicmath} % disable protrusion for tt fonts
}{}
\usepackage[margin=1in]{geometry}
\usepackage{hyperref}
\hypersetup{unicode=true,
            pdftitle={Problem Set 1},
            pdfauthor={Donovan Doyle},
            pdfborder={0 0 0},
            breaklinks=true}
\urlstyle{same}  % don't use monospace font for urls
\usepackage{graphicx,grffile}
\makeatletter
\def\maxwidth{\ifdim\Gin@nat@width>\linewidth\linewidth\else\Gin@nat@width\fi}
\def\maxheight{\ifdim\Gin@nat@height>\textheight\textheight\else\Gin@nat@height\fi}
\makeatother
% Scale images if necessary, so that they will not overflow the page
% margins by default, and it is still possible to overwrite the defaults
% using explicit options in \includegraphics[width, height, ...]{}
\setkeys{Gin}{width=\maxwidth,height=\maxheight,keepaspectratio}
\IfFileExists{parskip.sty}{%
\usepackage{parskip}
}{% else
\setlength{\parindent}{0pt}
\setlength{\parskip}{6pt plus 2pt minus 1pt}
}
\setlength{\emergencystretch}{3em}  % prevent overfull lines
\providecommand{\tightlist}{%
  \setlength{\itemsep}{0pt}\setlength{\parskip}{0pt}}
\setcounter{secnumdepth}{0}
% Redefines (sub)paragraphs to behave more like sections
\ifx\paragraph\undefined\else
\let\oldparagraph\paragraph
\renewcommand{\paragraph}[1]{\oldparagraph{#1}\mbox{}}
\fi
\ifx\subparagraph\undefined\else
\let\oldsubparagraph\subparagraph
\renewcommand{\subparagraph}[1]{\oldsubparagraph{#1}\mbox{}}
\fi

%%% Use protect on footnotes to avoid problems with footnotes in titles
\let\rmarkdownfootnote\footnote%
\def\footnote{\protect\rmarkdownfootnote}

%%% Change title format to be more compact
\usepackage{titling}

% Create subtitle command for use in maketitle
\newcommand{\subtitle}[1]{
  \posttitle{
    \begin{center}\large#1\end{center}
    }
}

\setlength{\droptitle}{-2em}

  \title{Problem Set 1}
    \pretitle{\vspace{\droptitle}\centering\huge}
  \posttitle{\par}
    \author{Donovan Doyle}
    \preauthor{\centering\large\emph}
  \postauthor{\par}
      \predate{\centering\large\emph}
  \postdate{\par}
    \date{1/30/2019}

\usepackage{booktabs}
\usepackage{longtable}
\usepackage{array}
\usepackage{multirow}
\usepackage[table]{xcolor}
\usepackage{wrapfig}
\usepackage{float}
\usepackage{colortbl}
\usepackage{pdflscape}
\usepackage{tabu}
\usepackage{threeparttable}
\usepackage{threeparttablex}
\usepackage[normalem]{ulem}
\usepackage{makecell}

\begin{document}
\maketitle

\section{Question 1}\label{question-1}

\textbf{A)} The minimum distance was 18. The maximum was 76, but it
missed. The mean was 36.90. The median was 37. \textbf{B)} The minimum
can't get lower than 18 because for each kick, you must also include the
distance of the end zone and the distance between the longsnapper and
the holder. The distance of the end zone is 10 yards, and the holder
will generally set up 7 yards behind the longsnapper, so any kick, even
from the 1 yard line, will be at least 18 yards. The maximum can be
explained because it occurred in the 30th minute, meaning right at
halftime. The Raiders probabyly thought they had a better chance trying
a kick than they did throwing a deep ball to the endzone.

\section{Question 2}\label{question-2}

The percentage of kicks made between 40 and 45 yards was 79.2\%, while
the percentage of kicks made above 45 was 64.4\%.

\section{Question 3}\label{question-3}

The make rate on grass was 82\%, while the make rate on turf was
slightly higher at 84\%. The difference is statistically significant at
a 99\% level when tested with no control variables. Even with control
variables added (I used GameMinute, ScoreDiff, and Distance), the
relationship is still statistically significant, but it is much smaller
than the 2\% in the raw make rates. 2\% is not the true effect of
surface, but there is an effect.

\section{Question 4}\label{question-4}

\textbf{A)} The correlation between grass and distance is -0.003. This
likely means when there's turf, a coach is very slightly more willing to
take a kick. \textbf{B)} The correlation between success and distance is
-0.337. This means the longer a kick, the less likely it is to be made.

\section{Question 5}\label{question-5}

\textbf{A)} Covariance of X and Y over the variance of X is the formula
for OVB. \textbf{B)} Distance is confirmed OVB, as it is independent
from X (grass) and impacts Y (success).

\section{Question 6}\label{question-6}

\textbf{A)} The Game Minute is not statistically significant compared to
the success rate, so there seems to be no significant correlation
between the two. There is no evidence of ``clutch'' kicking. \textbf{B)}
This corrects for skill of the kicker, isolating the effects of the Game
Minute on the success rate. The adjusted R-squared is now 0.1205,
greater than the old adjusted R-squared of 0.1139. \textbf{C)} It seems
kickers have been getting better over the years, as 2012-2015 all have
positive significant relationships with the make rate. This is in line
with most other athletic skills, as kickers get better over time, such
as a 4-minute mile supposedly being impossible or the average weight of
an offensive lineman increasing over the years.

\section{Question 7}\label{question-7}

\textbf{A)} There is an 88.87\% chance that Tucker makes the field goal,
given the conditions. \textbf{B)} Yes, this makes sense. I would assume
it would be slightly under his average, as the ScoreDiff and GameMinute
would skew his make rate down, given that you generally have to take
longer chances at GameMinute 30 and in that close of a game, meaning he
would have to kick longer field goals in that situation usually.

\section{Question 8}\label{question-8}

\textbf{A)} I got 88.87\% chance again. \textbf{B)} I think this may be
because R auto-runs a probablistic when the dependent variable is
between 0 to 1. \# Question 9 Clustered standard errors are used when
standard errors are correlated to each other in panel data. For example,
in our case, a kicker's standard error from one year to the next will be
related to each other, because if he is kicking in 2015 after playing in
2014, he likely had a pretty good year in 2015, or at least a good
enough year to not be benched. For this reason, I would cluster at the
kicker factor level, as that will likely be correlated with each other
more than at the year factor level.

\section{Question 10}\label{question-10}

\begin{table}[H]
\centering
\begin{tabular}{l|r|r}
\hline
Kicker & Make Rate & Kick Count\\
\hline
Akers & 0.81 & 336\\
\hline
Brown & 0.84 & 488\\
\hline
Bryant & 0.87 & 308\\
\hline
Crosby & 0.83 & 321\\
\hline
Dawson & 0.88 & 332\\
\hline
Gostkowski & 0.89 & 342\\
\hline
Gould & 0.85 & 329\\
\hline
Janikowski & 0.82 & 334\\
\hline
Vinatieri & 0.87 & 339\\
\hline
\end{tabular}
\end{table}

I took the kickers with the 9 most kicks in the time period, as volume
is important in my opinion, showing that the coach trusts them and they
never were beaten out. Of these kickers, Gostkowski has the highest make
rate. This provides solid evidence that Gostkowski is the best kicker of
the time period, although the numbers are very close. I think Brown
would also be a good choice, given his workload, his coaches must trust
him. In the regression that includes clustering kickers, which contains
controls for all of our quantifiable variables, Tucker and Vinatieri
have the strongest relationship between their kicking and success.

\section{Question 11}\label{question-11}

I would want 1) weather results, 2) playoff versus regular season
results, and 3) blocked kicks/messy snaps and holds vs clean kicks.
Controlling for weather results would positively affect the make rates,
as bad weather is bad for kicking, playoff results would help see the
``clutch'' gene better if it exists, and controlling for only clean
kicks would positively affect the make rates.

I used this data from Tony Crabtree that has weather by game for another
project, which would be helpful if I wanted to dig more:

\url{https://www.kaggle.com/tobycrabtree/nfl-scores-and-betting-data}

All of my work in R is contained here:
\url{https://github.com/donovandoyle/1042_PS1}


\end{document}
